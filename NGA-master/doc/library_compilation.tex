\documentclass[10pt, oneside, onecolumn]{article}
\usepackage{moreverb}   %for verbatim
\usepackage{graphicx}
\usepackage{amsmath, amssymb}
\usepackage[bf,hang, small]{caption}
\usepackage{epsfig}
\usepackage{psfrag}
\usepackage{color}

\setlength{\textwidth}{6.25in}
\setlength{\textheight}{8.5in}
\setlength{\topskip}{0mm}
\setlength{\evensidemargin}{0mm}
\setlength{\oddsidemargin}{0mm}
\renewcommand{\baselinestretch}{1.0}
\renewcommand{\thefootnote}{\fnsymbol{footnote}}
\graphicspath{{Figures/}}

\renewcommand{\sectionmark}[1]{\markboth{\sectionname\ \thesection. #1}{}}
\renewcommand{\sectionmark}[1]{\markright{\thesection\ #1}}
\renewcommand{\thesection}{\arabic{section}}

\begin{document}
  %\date{October 2005}
  \title{How-To Guide: \\
    Compiling the libraries required by NGA}
  \author{Ed Knudsen} 
  %\small{\textit{Department of Mechanical Engineering}} \\
  %\small{\textit{Stanford University}}
  %}
  \maketitle
  %\hrule
  %%%%%%%%%%%%%%%%%%%%%%%%%%%%%%%%%%%%%%%%%%%%%%%%%%%%%%%%%%%%%%%%%%%%%%%%%%%%%%%
  \section{Overview}
  %%%%%%%%%%%%%%%%%%%%%%%%%%%%%%%%%%%%%%%%%%%%%%%%%%%%%%%%%%%%%%%%%%%%%%%%%%%%%%%
  NGA requires 3 sets of external libraries.  These are:
  \begin{itemize}
    \item{Lapack \& Blas} 
    \item{FFTW}
    \item{Hypre}
  \end{itemize}
  Each of these libraries must be locally compiled on a machine before 
  NGA itself can be compiled.  Each is available for free download from the 
  web. As of the writing of this document, the versions of the libraries
  that are being used with NGA are:
  \begin{itemize}
    \item{Lapack \& Blas : 3.1.0} 
    \item{FFTW : 3.1.2}
    \item{Hypre : 2.0}
  \end{itemize}
  Unfortunately, the compilation of several of these libraries is not 
  intuitive.  This guide is intended to make the installation process
  go smoothly.   
  
  Additionally, to compile NGA a set of MPI libraries 
  must be available. Several implemetations of those libraries exist:
  MPICH, MPICH2, OPENMPI, ... This document explains how to compile the
  MPICH libraries. Currently, MPICH version 1.2.7p1 is being used with
  the code.
  
  While the Lapack, Blas, and FFTW libraries all work in serial, 
  Hypre is parallel computing software and therefore requires these
  MPI libraries during installation. The order of compilation is then:
  \begin{itemize}
    \item MPICH 
    \item Lapack \& Blas
    \item FFTW
    \item Hypre
  \end{itemize}
  
  % ---------------------------------------------------------------------------
  \section{MPICH}
  
  Download the file {\tt mpich-1.2.7p1.tar.gz} (or some equivalent) and 
  unpack it.  Installing MPICH involves four steps: 
  \begin{itemize}
    \item forcing to use certain compilers:
      \begin{verbatim}
	export CC=icc
	export CXX=icpc
	export F77=ifort
	export F90=ifort
      \end{verbatim}
    \item testing and configuring the system : {\tt ./configure
      --prefix=<directory where to install> --with-device=ch\_shmem}
    \item compiling the libraries: {\tt make}
    \item installing the libraries: {\tt make install}
  \end{itemize}
  We recommand installing the MPICH libraries in {\tt /opt/mpich/}.
  
  The catch is that when running the {\tt configure} program, the correct
  compiler executables such as (for the Intel compilers) {\tt icc} and 
  {\tt ifort} must be available in the environment so that the machine does 
  not default and compile with gcc or g77. To make sure that this is 
  the case, and if it is not already done, add the bin directory of
  your compiler to the environment path of the shell that you are
  working in. For example, in the bash 
  shell,
  \begin{verbatim}
    export PATH=$PATH:<path to Intel C/C++ compilers>/bin
    export PATH=$PATH:<path to Intel Fortran compilers>/bin
  \end{verbatim}
  If the path to the Intel compilers was not already included in your
  {\tt PATH} variable, it is also likely you will have to update
  another environmental variable:
  \begin{verbatim}
    export LD_LIBRARY_PATH=$LD_LIBRARY_PATH:<path to Intel C/C++ compilers>/lib
    export LD_LIBRARY_PATH=$LD_LIBRARY_PATH:<path to Intel Fortran compilers>/lib
  \end{verbatim}

  To make sure that the MPI wrappers were created for all compilers,
  look in the directory where you installed MPICH. In the {\tt bin}
  subdirectory, you should find {\tt mpicc}, {\tt mpicxx}, {\tt
  mpiCC}, {\tt mpif77}, {\tt mpif90}. If they are there then you are
  all set.  If not, go back to the {\tt configure} step and make sure that
  it recognizes the latest versions of your compilers.  

  % ---------------------------------------------------------------------------
  \section{Lapack \& Blas}

  Download and unpack the {\tt lapack-3.1.0.tgz} file (or something similar).
  In the directory, you should see a {\tt Makefile} and a {\tt
  make.inc} file, or at least a {\tt make.inc.example} file. If there
  is no {\tt make.inc} file, copy the {\tt make.inc.example} file to
  {\tt make.inc}. 
  
  Open the {\tt make.inc} file and change the Fortran compiler environment 
  variables to match the compiler on your local machine.  Lapack and 
  Blas are compiled using only Fortran, so you will not need to touch 
  any C/C++ compilers. For the Intel Fortran compiler, we recommand:
  \begin{verbatim}
    FORTRAN  = ifort
    OPTS     = -O3 -xN -ip
    DRVOPTS  = \$(OPTS)
    NOOPT    =
    LOADER   = ifort
    LOADOPTS = -O3 -xN -ip
  \end{verbatim}
  
  Usually {\tt\bf NGA} wants these libraries in the form {\tt libblas.a}
  and {\tt liblapack.a}. Change the names of the libraries in {\tt make.inc}:
  \begin{verbatim}
    BLASLIB      = ../../libblas.a
    LAPACKLIB    = liblapack.a
  \end{verbatim}

  Now you are ready to compile. The compilation is done in two steps:
  \begin{itemize}
    \item compiling Blas: {\tt make blaslib}
    \item compiling Lapack: {\tt make lapacklib}
  \end{itemize}
  
  Create the directory where you want to install the Lapack \& Blas
  libraries. We recommand installing the libraries in {\tt
  /opt/lapack}. Copy the two libraries ({\tt libblas.a} and {\tt
  liblapack.a}) to the install directory.

  \section{FFTW}

  Download {\tt fftw-3.1.2.tar.gz}, or something similar. Unpack it
  and you should get a series of files that include a {\tt
  configure} file. Again, you will want to make sure that the latest
  versions of the compilers you will use for {\tt\bf NGA}, such as
  {\tt ifort} and {\tt icc}, are available in the path so that FFTW
  can find them. Installing FFTW involves four steps: 
  \begin{itemize}
    \item forcing to use certain compilers:
      \begin{verbatim}
	export CC=icc
	export CXX=icpc
	export F77=ifort
      \end{verbatim}
    \item testing and configuring the system : {\tt ./configure
      --prefix=<directory where to install>}
    \item compiling the libraries: {\tt make}
    \item installing the libraries: {\tt make install}
  \end{itemize}
  We recommand installing the FFTW libraries in {\tt /opt/fftw/}. To
  make sure the FFTW libraries are correctly installed, look for the
  two files {\tt fftw3.h} and {\tt fftw3.f} in the {\tt include}
  directory where you installed FFTW.
  
  
  \section{Hypre}

  Download {\tt hypre-2.0.0.tar.gz}, unpack the file and go in the
  {\tt src} subdirectory. This is the easiest install of them all. The
  only thing unique about the Hypre installation is that instead of
  using {\tt ifort} and {\tt icc} to compile, Hypre will use {\tt
  mpif90} and {\tt mpicc}, so make sure the parallel compilers that
  you created using mpich are in your local path when you run these
  commands. Installing Hypre involves three steps:
  \begin{itemize}
    \item testing and configuring the system : {\tt ./configure
      --prefix=<directory where to install>}
    \item compiling the libraries: {\tt make}
    \item installing the libraries: {\tt make install}
  \end{itemize}
  We recommand installing the Hypre libraries in {\tt /opt/hypre/}. It
  is very important to download and install the latest version of
  Hypre, since {\tt\bf NGA} is not compatible with Hypre versions
  older than $2.0$.

\end{document}


